%%%%%%%%%%%%%%%%%%%%%%%%%%%%%%%%%%%%%%%%%%%%%%%%%%
\section{Skill Feats}
%%%%%%%%%%%%%%%%%%%%%%%%%%%%%%%%%%%%%%%%%%%%%%%%%%

Skill feats have an effect that scales up based on the number of ranks that you have in the Skill that they're linked to.

%%%
\featentry{Acquirer's Eye}{[Skill]}
%%%

You know what you want, even if other people have it right now. (This is a Skill feat that scales with your ranks in Appraise.)

\textbf{Benefits:} You gain +3 to your Appraise checks.

\textbf{4:} You automatically know if something is ordinary, masterwork, or magic when looking at it.

\textbf{9:} You can discover the properties of a magic item, including how to activate it (if appropriate) and how many charges are left (if it has them), with a successful Appraise check (DC item's caster level + 10) and 10 minutes of work.

\textbf{14:} Once per round as a free action, you can examine a magic item and attempt an Appraise check (DC item's caster level + 20) to determine its properties, including its functions, how to activate those functions (if necessary), and how many charges it has left (if it has charges).

\textbf{19:} You know what the most valuable piece of treasure is in any collection, such as the most valuable magic item an enemy is wearing or the most valuable object in a dragon's horde, just by looking at the collection. You automatically recognize an artifact when looking at it.

%%%
\featentry{Acrobatic}{[Skill]}
%%%
You can totally flip out and kill someone with your gymnastic prowess. (This is a Skill feat that scales with your ranks in Tumble.)

\textbf{Benefits:} You gain a +3 bonus to Tumble checks.

\textbf{4:} When using the Combat Expertise option, your dodge bonus to AC increases by +1. This further increases by +1 for every ten ranks of Tumble you have (+2 at 14, +3 at 24, and so on).

\textbf{9:} If an opponent attempts to bull-rush, overrun, or trample you, if you succeed on Tumble check of DC 25 + their base attack bonus, their movement continues in a straight line to the maximum allowed by their speed, you remain where you were, and you don't suffer from the effects of their bull-rush, overrun, or trample. If you fail, you provoke an attack of opportunity from that enemy.

\textbf{14:} If you succeed on a DC 40 Tumble check, you can move 10 feet when taking a 5-foot step.

\textbf{19:} If you succeed on a Tumble check against a DC of 30 + an opponent's base attack bonus, an action that would normally provoke an attack of opportunity doesn't.

%%%
\featentry{Alertness}{[Skill]}
%%%

Your ears are so sharp you probably wouldn't miss your eyes. (This is a Skill feat that scales with your ranks in Listen.)

\textbf{Benefits:} You gain a +3 bonus to Listen checks.

\textbf{4:} You can make a Listen check once a round as a free action. You don't take penalties for distractions on your Listen checks.

\textbf{9:} You gain blindsense to 60 feet. You don't take penalties for ambient noise, such as loud winds. Divide any distance penalties you take on Listen checks by two.

\textbf{14:} You gain blindsight to 120 feet.

\textbf{19:} You can hear through magical silence and similar effects, but you take a -20 penalty on your check. Divide any distance penalties you take on Listen checks by five.

%%%
\featentry{Animal Affinity}{[Skill]}
%%%

You're one of those people animals just won't leave alone for no apparent reason. (This is a Skill feat that scales with your ranks in Handle Animal.)

\textbf{Benefits:} You gain the wild empathy ability, with your check equal to your character level plus your Charisma modifier plus any other applicable bonuses. If you already have wild empathy, or later gain it from another source, you gain a +3 bonus on Handle Animal checks.

\textbf{4:} You can handle an animal as a free action, and push it as a move action.

\textbf{9:} You gain the benefits of speak with animals permanently as an extraordinary ability. The DCs for you to rear and train creatures are halved.

\textbf{14:} With a DC 30 Handle Animal check, you can use a mass version of charm animal as a spell-like ability, with save DC equal to 10 + half your character level + your Cha modifier and effective caster level equal to your bonus on Handle Animal checks.

\textbf{19:} You can summon animals to your aid. Choose an animal with a CR equal to or less than your character level, and make a Handle Animal check at a DC of 25 + your character level. If you succeed, you summon a number of animals depending on how much the animal's CR is less than your character level for an hour. You can't use this ability again until any animals you've summoned with it have unsummoned or you've dismissed them.

\begin{basictable}{Animal Affinity Summoning}{l l}
\textbf{CR} & \textbf{Number Appearing}\\
Level - 1 & 1\\
Level - 2 & d3\\
Level - 3 & d4\\
Level - 4 & d6\\
Level - 5 & d8\\
Level - 6 & d10\\
Level - 7 & 2d6\\
Level - 8 & 3d6\\
Level - 9 & 3d10\\
Level - 10 & 3d6+10\\
Level - 11 & 3d10+15\\
Level - 12 & 40\\
Level - 13 & 50\\
Level - 14 & 60\\
Level - 15 & 80\\
Level - 16 & 100\\
Level - 17 & 150\\
Level - 18 & 200\\
Level - 19 & 300\\
\end{basictable}

%%%
\featentry{Battlefield Surgeon}{[Skill]}
%%%

You like to cut people open with a saw. But it's good for them. Seriously. (This is a Skill feat that scales with your ranks in Heal.)

\textbf{Benefits:} You gain +3 to your Heal checks.

\textbf{4:} You can make first aid, treat poison, and treat wound checks as move actions.

\textbf{9:} For every 5 points your Heal check exceeds the DC for long term care, your patients recover another +100\% faster. For instance, if your Heal check result is 23, your patients would heal at thrice the normal rate.

\textbf{14:} If you operate on a patient for a minute, they regain hit points equal to your Heal check result. You also may, with a DC 30 check, instead of healing hit point damage, cure any condition that \linkspell{Heal} could, reattach severed limbs, or repair ruined organs. Patients under your long-term care heal permanent ability drain as if it was ability damage.

\textbf{19:} With one hour of work, 50,000 gp worth of materials (which are consumed in the process), and a DC 40 Heal check, you can restore a creature that died within the last twenty-four hours to life. The subject's soul must be free and willing to return for the effect to work.

Your patient comes back from the dead with full hit points and all conditions cured, with no loss of prepared spells or spell slots. You must have a reasonably intact body to work with, though you can reattach missing parts, if you have them, and repair organs.

You can restore life to a creature killed by a death effect. You cannot affect constructs, elementals, outsiders (except native outsiders), and undead creatures, nor creatures that have died of old age.

%%%
\featentry{Combat Casting}{[Skill]}
%%%

Having a sword sticking out of your chest doesn't noticeably impede your ability to do—well, just about anything. (This is a Skill feat that scales with your ranks in Concentration.)

\textbf{Benefits:} You gain +3 to your Concentration checks.

\textbf{4:} You can take 10 on Concentration checks and caster level checks.

\textbf{9:} You may maintain concentration on a spell as a move action (DC 25 + spell level). If you beat the DC by 10 or more, you can maintain concentration as a swift action. If you fail your check, you lose concentration.

\textbf{14:} If you would be nauseated, you're sickened instead.

\textbf{19:} All Concentration DCs are halved for you.

%%%
\featentry{Con Artist}{[Skill]}
%%%

You can fool some of the people, all of the time. (This is a Skill feat that scales with your ranks in Bluff.)

\textbf{Benefits:} You gain a +3 bonus to Bluff checks.

\textbf{4:} Magic effects that would detect your lies or force you to speak the truth must succeed on a caster level check with DC equal to 10 plus your bonus on Diplomacy checks or fail.

\textbf{9:} Divination magic used on you detects a false alignment of your choice. You can present false surface thoughts to detect thoughts and similar effects, changing your apparent Intelligence score (and thus your apparent mental strength) by as much as 10 points and can place any thought in your "surface thoughts" to be read by such spells or effects.

\textbf{14:} If you beat someone's Sense Motive check by 25, you can instill a suggestion in them, as the spell. This suggestion lasts for one hour for each of your character levels.

\textbf{19:} You are protected from all spells and effects that detect or read emotions or thoughts, as by mind blank.

%%%
\featentry{Cryptographer}{[Skill]}
%%%

You're good at reading things no one intended you to. (This is a Skill feat that scales with your ranks in Decipher Script.)

\textbf{Benefits:} You gain +3 to your Decipher Script checks.

\textbf{4:} You can decipher a written spell (like a scroll) without using read magic, if you succeed on a Decipher Script check of DC 20 + the spell's level. You can try once per day on any particular written spell.

\textbf{9:} You don't trigger written magic traps (like explosive runes or symbols) by reading them. You can disable them with Decipher Script as if you were using Disable Device. You can read the material hidden by a secret page with a DC 25 Decipher Script check.

\textbf{14:} When you cast a spell from a scroll, the spell's save DC is equal to 10 + the spell's level + your Intelligence modifier + any other applicable bonuses, and its caster level is equal to your character level, plus other applicable bonuses.

\textbf{19:} Reading text using Decipher Script is a free action for you. You may disable written magical traps as a swift action, and you can cast 5th-level or lower spells from scrolls as a swift action.

%%%
\featentry{Deft Fingers}{[Skill]}
%%%

Your amazing manual dexterity is the talk of princes and princesses. (This is a Skill feat that scales with your ranks in Sleight of Hand.)

\textbf{Benefits:} You gain a +3 bonus on your Sleight of Hand checks.

\textbf{4:} If you draw a hidden weapon and attack with it in the same round, your opponent loses their Dexterity bonus to AC against your first attack with that weapon that round.

\textbf{9:} You can make an adjacent creature or object your size or smaller 'disappear' with your legerdemain. If you succeed on a DC 30 Sleight of Hand check as a standard action, your target can make a Hide check, or you can make the Hide check for them or it. As usual, you can hide larger creatures or objects by taking a -20 cumulative penalty for each size category larger they are than you.

\textbf{14:} With a DC 30 Sleight of Hand check, you can use shrink item as a spell-like ability.

\textbf{19:} With a DC 40 Sleight of Hand check, you can use teleport object as a spell-like ability. You can also retrieve items placed in the Ethereal Plane using teleport object. With a DC 40 Sleight of Hand check, you can use instant summons as a spell-like ability without requiring arcane mark, but you may only designate one item at a time.

%%%
\featentry{Detective}{[Skill]}
%%%

You're good at finding things out just by conversing with townsfolk. (This is a Skill feat that scales with your ranks in Gather Information.)

\textbf{Benefits:} You gain a +3 bonus on your Gather Information checks.

\textbf{4:} Your ability to pick up on the social context aids you in establishing rapport. After succeeding on a Gather Information check, you gain a +2 bonus to Knowledge checks, Sense Motive checks, and checks for Cha-based skills in the same milieu.

\textbf{9:} With 2d6 hours of research, you can study a specific topic, such as a particular location or a well-known local monster, and substitute a Gather Information check for any Knowledge checks pertaining to the topic. You need access to local informants, a library, scholars, or other appropriate sources to use this ability.

\textbf{14:} You can gain the benefits of legend lore with a DC 30 Gather Information check. If you have the person or thing at hand, or are in the place, this takes a day; otherwise, it consumes the time as normal for legend lore. You need access to individuals or resources with relevant knowledge to use this ability.

\textbf{19:} With a DC 40 Gather Information check and 1d4+1 days of talking to people, you can either find an answer to any question you can pose in ten words or less, or find out where you need to go to get the answer. You need access to individuals or resources with relevant knowledge to use this ability.

%%%
\featentry{Dreadful Demeanor}{[Skill]}
%%%

People know you're a badass motherfvcker the instant you enter the room. (This is a Skill feat that scales with your ranks in Intimidate.)

\textbf{Benefits:} You gain +3 to your Intimidate checks.

\textbf{4:} You can demoralize an opponent as a move action.

\textbf{9:} Opponents you've demoralized remain shaken until they lose sight of you.

\textbf{14:} Opponents who would be panicked because of your fear effects are cowered instead.

\textbf{19:} Any time you confirm a critical hit in melee, your target is cowered. This is a fear effect.

%%%
\featentry{Expert Counterfeiter}{[Skill]}
%%%

You aren't a common forger, you're an artiste. (This is a Skill feat that scales with your ranks in Forgery.)

\textbf{Benefits:} You gain a +3 bonus to Forgery checks.

\textbf{4:} When creating a forgery, you roll twice and take the better result.

\textbf{9:} In situations where you can present a legal document of some sort, you can substitute a Forgery check for a Bluff, Diplomacy, or Intimidate check.

\textbf{14:} You can purchase items with counterfeit bills of exchange, falsified credit vouchers, and the like. You can acquire any item available through the gold economy in this method. Normally, your counterfeits are so good they don't provoke suspicion, but if someone examines them, they must still beat you in an opposed Forgery check to recognize they're not the real thing.

\textbf{19:} You can duplicate a scroll with eight hours of work and a Forgery check against DC 35 + the spell's level. The duplicate functions in all manners like the original scroll. You must have appropriate materials on hand for scribing the scroll, and if the spell requires XP or expensive material components, you must provide the requisite components or make up the XP cost in materials.

%%%
\featentry{Ghost Step}{[Skill]}
%%%

You might as well be incorporeal for all the noise you make. (This is a Skill feat that scales with your ranks in Move Silently.)

\textbf{Benefits:} You gain +3 to your Move Silently checks.

\textbf{4:} Anyone attempting to use Survival to track you must beat you in an opposed check against Move Silently.

\textbf{9:} Creatures with blindsense, blindsight, tremorsense, or similar abilities do not automatically detect your presence, but must succeed on a Listen check, opposed by your Move Silently check, to notice you.

\textbf{14:} With success on a DC 30 Move Silently check as a standard action, you can control ambient sounds within 30 feet of yourself for a round. You can specifically duplicate any effect from control sound (XPH), silence, or ventriloquism, and in general can make sound you've heard come from any part of the area, displace sounds in the area, or suppress any sounds or sounds. Also, if you take a -10 DC penalty on your Move Silently check, anyone within 30 feet of you can substitute your check result for their own.

\textbf{19:} You're so quiet that people don't even remember you when you're standing right next to them. Your opponents count as flat-footed whenever you attack them.

%%%
\featentry{Investigator}{[Skill]}
%%%

You have an eye for detail and so much patience that going through a 100' by 100' room inch-by-inch doesn't even try it. (This is a Skill feat that scales with your ranks in Search.)

\textbf{Benefits:} You can use Search to find traps like a character with trapfinding. If you already have that ability, you gain +3 to your Search checks. Search is always a class skill for you.

\textbf{4:} You can Search a 10' by 10' area with a full-round action.

\textbf{9:} You automatically sense any active magic effects in an area you search. If you succeed on a DC 20 Search check, you can determine their number, strength, and school, as if using detect magic.

\textbf{14:} You can Search objects or areas within 30 feet of yourself. You can make a Search check as a swift action.

\textbf{19:} You have an intuitive sense for hidden things. Anytime something that someone has hidden is within 60 feet of you, you know it; if there are multiple things, you know how many. However, you must still make Search checks as normal to locate them.

%%%
\featentry{Item Master}{[Skill]}
%%%

You make magic items do things you want. (This is a Skill feat that scales with your ranks in Use Magic Device.)

\textbf{Benefits:} You gain a +3 bonus to Use Magic Device checks.

\textbf{4:} You don't suffer mishaps with magic items.

\textbf{9:} When rolling Use Magic Device checks or random effects from magic items, you may roll twice and take the better result.

\textbf{14:} With a swift action and a successful Use Magic Device check against a DC of 30 + the item's caster level, you can gain the benefits of a slotted magic item without needing to have a slot available (for instance, a third ring on your finger) for one round.

\textbf{19:} When you activate a wand or staff, you can substitute a spell slot instead of using a charge. The spell slot must be one you have not used for the day, though you may lose a prepared spell to emulate a wand charge (you may not lose prepared spells from your school of specialty, if any). The spell slot lost must be equal to or higher in level than the spell stored in the wand, including any level-increasing metamagic enhancements. When using spell trigger, spell completion, or other consumable magic items, if you succeed on a Use Magic Device check of 40 + the caster level of the item as a swift action, the item or charges thereof are not consumed.

%%%
\featentry{Leap of the Heavens}{[Skill]}
%%%

You jump good. (This is a Skill feat that scales with your ranks in Jump.)

\textbf{Benefits:} The DCs for your jumps don't double if you fail to get a running start, and if you do, you get a +4 bonus on the check. You can hop up (see the Jump skill description) onto any object shorter than your height without a Jump check.

\textbf{4:} You don't take falling damage, though you can still take damage if something falls on you.

\textbf{9:} You ignore the effects of difficult terrain on your movement speed, skill checks, and ability to charge.

\textbf{14:} If you succeed a DC 40 Jump check as a swift action, you gain the benefits of fly for one round.

\textbf{19:} For every five ranks you have in Jump, your movement speeds increase by 10'. (This also increases your Jump checks, as usual.)

%%%
\featentry{Legendary Wrangler}{[Skill]}
%%%

No one can tell where you end and your ropes begin. (This is a Skill feat that scales with your ranks in Use Rope.)

\textbf{Benefits:} You gain a +3 bonus to Use Rope checks and proficiency with the bolas, net, and whip.

\textbf{4:} You can use a rope as if it was a bolas or whip, and you can substitute your ranks in Use Rope for your Base Attack Bonus for combat maneuvers made with it. You can also use it as a net, replacing the normal DC 20 Escape Artist check for someone entangled with it with your Use Rope check. You can throw a grappling hook, tie a knot, tie a special knot, or tie a rope around yourself one-handed as a move action. You don't provoke attacks of opportunity for using Use Rope.

\textbf{9:} You can use a rope, whip, grappling hook, or similar item to manipulate any item within 30 feet of yourself as easily as if it was in your hands; you can also make disarm, entangling (as if with a net), and trip attempts with it. You can move around on ropes and similar structures, like webs, as easily as you can on the ground.

\textbf{14:} With a DC 30 Use Rope check, you can use animate rope as a spell-like ability; you can use any ability you can with an ordinary rope with an animated rope.

\textbf{19:} You can manipulate items out to 60 feet with ropes and similar items. You can use ropes for the grab on and hold down grapple maneuvers. When using combat maneuvers with ropes, you can replace the relevant check (disarm, grapple, trip, etc.) with a Use Rope check.

%%%
\featentry{Magical Aptitude}{[Skill]}
%%%

You're crazy good at manipulating magic. (This is a Skill feat that scales with your ranks in Spellcraft.)

\textbf{Benefits:} You gain a +3 bonus on Spellcraft checks.

\textbf{4:} When counterspelling, you may use a spell of the same school that is one or more spell levels higher than the target spell.

\textbf{9:} You can dismiss a spell as a free action. You can redirect a spell as a move action, if it normally requires a standard action, or a swift action, if it normally takes a move action. You gain a +3 bonus on dispel checks.

\textbf{14:} You can counter a spell as an immediate action.

\textbf{19:} You automatically know which spells or magic effects are active on upon any individual object you see, as if you had greater arcane sight active on yourself.

%%%
\featentry{Many-Faced}{[Skill]}
%%%

You change identities so often even you don't remember what you look like anymore. (This is a Skill feat that scales with your ranks in Disguise.)

\textbf{Benefits:} You gain +3 to your Disguise checks.

\textbf{4:} When creating a disguise, you roll twice and take the better result.

\textbf{9:} You can use Magic Aura as a spell-like ability at will, with a caster level equal to your character level and a save DC of 10 + half your character level + your Cha modifier.

\textbf{14:} You can create a disguise as a full-round action, but you take a -10 penalty to your Disguise check. You can't be under direct observation while doing this, but you can use Bluff to create a diversion to allow you to change guises, as for the Hide skill.

\textbf{19:} You can choose an appearance that anyone viewing you with scrying or other divination magic sees instead of your "real" appearance. Even someone who benefits from true seeing must succeed on a caster level check (DC 11 + your ranks in Disguise) to penetrate the illusion.


%%%
\featentry{Mounted Combat}{[Skill]}
%%%

You are at your best when fighting with an ally that you are sitting on. This is a Skill Feat that scales with your ranks in Ride.

\textbf{Benefits:} Once per turn, you may attempt to negate an attack that hits your mount by making a Ride skill check with a DC equal to the AC that the attack hit. Attacks that do not require an attack roll cannot be negated in this way.

\textbf{4:} While Mounted, you may take a charge attack at any point along your mount's movement, so long as your mount is moving in a straight line up to the point of your attack.

\textbf{9:} You suffer no penalty to your ride or handle animal skill checks when training or riding unusual mounts such as magical beasts or dragons.

\textbf{14:} You may use your Ride Check in place of your mount's Balance, Jump, Climb, or Reflex Saving Throws.

\textbf{19:} Any time a spell effect would target your mount, you may elect to have it target you instead. Any time a spell effect would target you, you may elect to have it affect your Mount instead.

%%%
\featentry{Natural Empath}{[Skill]}
%%%

You read people like books. (This is a Skill feat that scales with your ranks in Sense Motive.)

\textbf{Benefits:} You gain a +3 bonus to Sense Motive checks.

\textbf{4:} You can quickly size up potential opponents. If you succeed on a Sense Motive check as a free action, opposed by their Bluff, you can tell if they're an even match (their CR equals your character level), an easy challenge (their CR is 1-3 less than your level), irrelevant (their CR is 4 or more less than your level), stronger (their CR is 1-3 higher than your level), or overwhelmingly powerful (their CR is 4 or more higher than your level). You can use this ability once on a particular creature every 24 hours.

\textbf{9:} If you succeed on a Sense Motive check, opposed by Bluff, you know your opponent's alignment. If you beat their Bluff by 20 or more, you can read their surface thoughts, as if during the third round of detect thoughts.

\textbf{14:} You have an uncanny intuition for when people are interested in you. Any time someone uses a remote spell or effect, like scrying, to examine you, you know you're under observation and if you make a Sense Motive check that beats their Bluff check, you know some details about them: if you've met them before, you recognize them, but if not, you get a basic idea of their reasons for their interest in you. Similarly, if you use Sense Motive on someone influenced by an enchantment effect, you can find out who created the effect with a Sense Motive check opposed by the controller's Bluff, getting the same information.

\textbf{19:} You know what people are going to do before they do. Any time someone you're aware of attacks you, make a Sense Motive check opposed by their Bluff: if you succeed, you get a free surprise round.

%%%
\featentry{Persuasive}{[Skill]}
%%%

When you tell you people something that contradicts the evidence of their own eyes, they believe you. (This is a Skill feat that scales with your ranks in Diplomacy.)

\textbf{Benefits:} You gain a +3 bonus to Diplomacy checks.

\textbf{4:} Your words can stop fights before they start. Any creature that can hear you speak must make a Will save (DC 10 + half your character level + your Cha modifier) or it can't attack you directly; however, you aren't protected from its area or effect spells, or similar abilities. Any creature that succeeds on its save is immune to this ability for 24 hours. You may use nonattack spells or otherwise act, but if you attack the creature or its allies, it may attack you. This is a mind-affecting, language-dependent charm effect.

\textbf{9:} You can fascinate creatures with your silver tongue. You can affect as many HD of creatures as your bonus on Diplomacy checks; any creature that fails a Will save (DC 10 + half your character level + your Cha modifier) becomes fascinated. If you use this ability in combat, each target gains a +2 bonus on its saving throw. If the spell affects only a single creature not in combat at the time, the saving throw has a penalty of -2. While a subject is fascinated by this spell, it reacts as though it were two steps more friendly in attitude, allowing you to make a single request of an affected creature. The request must be brief and reasonable. Even after the spell ends, the creature retains its new attitude toward you, but only with respect to that particular request. A creature that fails its saving throw does not remember that you enspelled it. 

\textbf{14:} You can influence even hostile creatures into talking things over with you. With a DC 30 Diplomacy check, you can use a language-dependent version of charm monster as a spell-like ability, with save DC equal to 10 + half your character level + your Cha modifier; this is a mind-affecting charm effect.

\textbf{19:} You can convince an entire group of enemies to listen to you. If you succeed on a DC 40 Diplomacy check, your charm monster ability improves to mass charm monster, with a caster level equal to your bonus on Diplomacy checks.

%%%
\featentry{Professional Luddite}{[Skill]}
%%%
You've learned to break machines because you're an antitechnology fanatic—or maybe you just work for the local protection racket. (This is a Skill feat that scales with your ranks in Disable Device.)

\textbf{Benefits:} You can use Disable Device on magic traps like a character with trapfinding. If you already have that ability, you gain +3 to your Disable Device checks. Disable Device is always a class skill for you.

\textbf{4:} You can use your Dexterity modifier instead of your Intelligence modifier for Disable Device checks. Darkness and blindness do not hinder your ability to disable devices.

\textbf{9:} You can reduce the amount of time required to disable a device. For each multiple of 10 you beat the required DC, you can decrease the time required from 2d4 rounds to 1d4 rounds to 1 round to a standard action to a move-equivalent action to a free action.

\textbf{14:} You can use Disable Device to end any persistent effect or area spell effect as if it was a magic trap, but the DC is 25 + twice the spell's level.

\textbf{19:} As an attack action, you can disable magic items. You must succeed on a melee touch attack roll for attended objects. Make a Disable Device check against a DC of 15 + the item's caster level: if your check succeeds, the item must make a Will save against a DC of 10 + half your caster level or be turned into a normal item, and even if it saves, its magical properties are suppressed for 1d4 rounds.

%%%
\featentry{Sharp-Eyed}{[Skill]}
%%%

Nothing escapes you. (This is a Skill feat that scales with your ranks in Spot.)

\textbf{Benefits:} You gain a +3 bonus to Spot checks.

\textbf{4:} You can make a Spot check once a round as a free action. You don't take penalties for distractions on your Spot checks.

\textbf{9:} As a move action, you can make a Spot check against a DC of an opponent's Armor Class: if you succeed, you can ignore their Armor and Natural Armor bonus to AC for the next attack you make against them. If you accept a -20 penalty to your check, you can attempt this check as a swift action. Divide any distance penalties you take on Spot checks by two.

\textbf{14:} If you beat an opponent's Hide check with a Spot check at a -10 penalty, you can ignore concealment. If you beat their Hide check at a -30 penalty, you can ignore total concealment.

\textbf{19:} You can see through solid objects, but you take a -20 penalty on your Spot check for each 5'. Divide any distance penalties you take on Spot checks by five.

%%%
\featentry{Slippery Contortionist}{[Skill]}
%%%

Your childhood nickname was "Greasy the Pig", but now people call you "The Great Hamster". (This is a Skill feat that scales with your ranks in Escape Artist.)

\textbf{Benefits:} You gain +3 to your Escape Artist checks.

\textbf{4:} While squeezing into a space at least half as wide as your normal space, you may move your normal speed and you take no penalty to your attack rolls or AC for squeezing.

\textbf{9:} You can squeeze through a tight space or an extremely tight space as a full-round action, but you take a -10 penalty to your Escape Artist check. Opponents grappling you don't get positive size modifiers added to their grapple bonus when you use Escape Artist to try to break their hold.

\textbf{14:} If you succeed on a DC 30 Escape Artist check, you can ignore magical effects that impede movement as if you were under the effects of freedom of movement for one round; this is not an action. You can also slip through a wall of force or similar barrier with a DC 40 check.

\textbf{19:} You can make an Escape Artist check instead of a saving throw for any effect that would keep you from taking actions. (This does not help against effects that don't allow a saving throw.)

%%%
\featentry{Steady Stance}{[Skill]}
%%%

You can fight just about anywhere. (This is a Skill feat that scales with your ranks in Balance.)

\textbf{Benefits:} You gain a +3 bonus to your Balance checks.

\textbf{4:} If an effect would knock you prone, if you succeed on a DC 20 Balance check, you remain standing.

\textbf{9:} If your opponent is balancing, you gain a +3 dodge bonus to AC against their attacks unless they succeed at beating you in an opposed Balance check.

\textbf{14:} All Balance DCs are halved for you.

\textbf{19:} You never suffer any impairment or damage from anything you're standing on, whether it's molten lava, a cloud, or even another creature. Ambient conditions, such as lighting or weather, can still impair you.

%%%
\featentry{Stealthy}{[Skill]}
%%%

If someone sees you, you have to kill them. (This is a Skill feat that scales with your ranks in Hide.)

\textbf{Benefits:} You gain a +3 bonus to your Hide checks.

\textbf{4:} You can Hide as a free action after attacking, and snipe with melee attacks (or ranged attacks from closer than 10').

\textbf{9:} A constant nondetection effect protects you and your equipment, with an effective caster level equal to your ranks in Hide.

\textbf{14:} You can attempt to Hide even when under direct observation, but you take the usual -20 penalty to your check.

\textbf{19:} Even opponents who can see you have trouble locating you. If they succeed at beating your Hide check with Spot (and thus can see you), they have a 50\% concealment miss chance when attacking you, which decreases by 5\% for each point they beat your Hide DC.

%%%
\featentry{Super Scaler}{[Skill]}
%%%

You stick to walls like glue. Or something. (This is a Skill feat that scales with your ranks in Climb.)

\textbf{Benefits:} You gain +3 to your Climb checks.

\textbf{4:} You have the Edge on an opponent if you're on higher ground. You don't lose your Dexterity bonus to AC while climbing, nor do you need to make a Climb check if hit while climbing.

\textbf{9:} You gain a climb speed equal to your base land speed, with the attendant benefits.

\textbf{14:} All Climb DCs are halved for you. While climbing, you can substitute a Climb check result for a Reflex saving throw, once per round.

\textbf{19:} While climbing, as an immediate action, you can add the result of a Climb check as a dodge bonus to your Armor Class against a single attack.

%%%
\featentry{Swim Like a Fish}{[Skill]}
%%%

You're at least as home in the water as you are on land. (This is a Skill feat that scales with your ranks in Swim.)

\textbf{Benefits:} You gain +3 to your Swim checks.

\textbf{4:} You gain a swim speed equal to your base land speed, with the attendant benefits. You don't take armor check penalties to your Swim checks.

\textbf{9:} You can breathe water, and you can attack through water as if under the effects of freedom of movement.

\textbf{14:} While under water, you can substitute Swim checks for Reflex saves, and you gain a +4 bonus to attack and damage rolls.

\textbf{19:} As a swift action, you can add your ranks in Swim as a dodge bonus to your Armor Class while under water.

%%%
\featentry{Track}{[Skill]}
%%%

You feel at home no matter where you are. (This is a Skill feat that scales with your ranks in Survival.)

\textbf{Benefits:} You can follow tracks using Survival, as the Track and Legendary Trackers feats.

\textbf{4:} You can identify the race/kind of creatures from their tracks.

\textbf{9:} You can move through or over difficult natural terrain without being slowed, taking nonlethal damage, or suffering other impairment. You take no penalties for moving your speed when tracking, and only -10 when moving double your speed. You can track subjects protected by pass without trace or similar spells at a -20 penalty.

\textbf{14:} You can track through the Astral Plane with a DC 35 Survival check. You can determine the destination of a teleport spell when standing at the point of departure with a DC 40 Survival check; if you have teleport or a similar spell, you can follow as if you had seen the destination once.

\textbf{19:} You're immune to natural planar effects as if you had planar tolerance always active.
