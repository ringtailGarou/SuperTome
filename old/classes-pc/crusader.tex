%%%%%%%%%%%%%%%%%%%%%%%%%%%%%%%%%%%%%%%%%%%%%%%%%%
\classentry{Crusader}
%%%%%%%%%%%%%%%%%%%%%%%%%%%%%%%%%%%%%%%%%%%%%%%%%%
\tagline{"All of my organs yearn to see this task done."}

The Crusader is a martial adept who is dedicated to and driven by some sort of cause. It may be good or evil, law or chaos, racial purity, religious conviction, or something else entirely. The Crusader trains hard and learns many forms of combat, but while in the heat of battle they are driven by inspiration. Some say they are guided by the divine, others that they simply train with too many forms to have a reliable battle order when blades are drawn.

Many Crusaders are members of specific schools of martial adeptness. These schools stress martial virtues, train with specific weaponry, and have identifiable styles. A character familiar with martial adepts can easily identify a Crusader's school simply by watching them fight. Your campaign world may have special schools of martial thought in it, discuss it with your Dungeon Master. The assumed schools of martial combat are White Raven (which uses the spiked chain, three section staff, and full blade), Devoted Dragon (which uses the shard sword, the skip rock, and the thin blade), Earth Spirit (which uses the war axe, the dire pick, and the ribbon dagger), Storm Guardian (which uses the pincer staff, the harpoon, and the net), and Amethyst Mind (which uses the maul, the mancatcher, and the shotput).

\textbf{Playing A Crusader:} Many of a Crusader's abilities are triggered off of Charisma, and until the Crusader gains a few levels they will not have much in the way of abilities that function outside of melee range. Crusaders therefore are usually made with Charisma and Strength as primary attributes. In combat, a Crusader is fairly resilient and should usually find themselves pushed to the front line. Out of combat, Crusaders generally contribute to the party with social and leadership abilities.

\textbf{Alignment:} Crusaders can and do fight for any cause, more often for ideology than for pure mercenary interest. A Crusader can be of any alignment, though the class implies a certain amount of fanaticism. The Crusader detects their alignment as if they were an Undead creature (ex.: with a moderate aura at level 3).

\textbf{Races:} Any

\textbf{Starting Gold:} 8d4x10 gp (200 gold)

\textbf{Starting Age:} As Fighter.

\textbf{Hit Die:} d10

\textbf{Class Skills:} The Crusader's class skills are \linkskill{Balance}, \linkskill{Climb}, \linkskill{Craft}, \linkskill{Diplomacy}, \linkskill{Disguise}, \linkskill{Gather Information}, \linkskill{Handle Animal}, \linkskill{Hide}, \linkskill{Intimidate}, \linkskill{Jump}, \linkskill{Knowledge} (Any), \linkskill{Listen}, \linkskill{Move Silently}, \linkskill{Perform}, \linkskill{Profession}, \linkskill{Ride}, \linkskill{Search}, \linkskill{Sleight of Hand}, \linkskill{Spot}, \linkskill{Survival}, \linkskill{Swim}.

\textbf{Skills/Level:} 4 + Intelligence Bonus

\goodbab{}
\goodfor{}
\goodref{}
\goodwil{}

\begin{extraclasstable}{\multicolumn{1}{p{2cm}}{\textbf{Maneuvers Readied}}}
\levelone{Stance, Crusader Maneuvers, Furious Counterstrike & 6 (2)}
\leveltwo{Steely Resolve, Banner of Conviction & 6 (2)}
\levelthree{Indomitable Soul & 6 (2)}
\levelfour{Stance, Mettle & 6 (2)}
\levelfive{Zealous Surge & 6 (2)}
\levelsix{Smite, Interrupting Fury & 6 (2)}
\levelseven{Stance	& 8 (3)}
\leveleight{Never Give Up & 8 (3)}
\levelnine{Very Persuasive & 8 (3)}
\levelten{Stance, Never Surrender & 8 (3)}
\leveleleven{So Much Smiting & 8 (3)}
\leveltwelve{Landlord & 10 (4)}
\levelthirteen{Stance, So Much Zealotry & 10 (4)}
\levelfourteen{Improved Mettle & 10 (4)}
\levelfifteen{Keep Fighting & 10 (4)}
\levelsixteen{Stance, Leadership & 10 (4)}
\levelseventeen{-- & 12 (5)}
\leveleighteen{Weapon Focus & 12 (5)}
\levelnineteen{Stance & 12 (5)}
\leveltwenty{World Conquest & 12 (5)}
\end{extraclasstable}

\classfeatures

\textbf{Weapon and Armor Proficiency:} A Crusader is proficient with all simple and martial weapons. Crusaders are also proficient with light, medium, and heavy armor as well as with shields and great shields. If the Crusader is a member of a church or school, the character gains proficiency with their favored weapons, even if those weapons are exotic.

\textbf{Crusader Maneuvers (Ex):} Crusaders can use special combat techniques called "maneuvers". Maneuvers come as Strikes (which are standard actions), Counters (which are Immediate Actions), and Boosts (which are Swift Actions that normally have 1 round durations). A Crusader can use each available maneuver once before it becomes expended, however they get new available maneuvers every round (which may include maneuvers expended the previous round). Initiating a maneuver is an extraordinary ability. At first level, a Crusader knows 6 maneuvers, and learns two additional maneuevers every time they gain a level. A Crusader may also learn new maneuvers by undergoing a one day "training montage" with a teacher who knows the maneuver in question or an exhaustive written description of how the maneuver is performed. In order to learn a new maneuver, the character must meet the level requirement of it.

If a Crusader's maneuvers or stances call on an opponent to make a saving throw, the DC is 10 + half level + Charisma modifier. If the Crusader is not in a battle and has not rolled initiative, they can be expected to be able to use any of their maneuvers given some time to psyche themselves up. Figure about one full round action to choose a specific maneuver if for some reason it is important enough to count time in rounds but not stressful enough that the character would roll initiative or otherwise be under threat.

\textbf{Maneuvers Readied and Available:} Though a Crusader can use their abilities an unlimited number of times per day, they have a limited number of Maneuvers that are "ready" at any given time. Further, the Crusader relies upon the flash of divine or fanatical inspiration in battle and does not have full control over what maneuvers can be used at any given moment. The Crusader can change what maneuvers they have readied with five minutes of prayer and planning, but each round the Crusader randomly determines which maneuvers are "available" from among their readied maneuvers. The maneuvers are determined randomly at the beginning of the Crusader's turn, and continue to be available until the beginning of the character's next turn.

At first level, a Crusader can ready 6 maneuvers and two are available each round. The character may not ready more or less maneuvers than their maximum, and all readied maneuvers must be different. At 7th level, the number of maneuvers the Crusader readies increases to 8 and the number that are available each round increases to 3. These increase to 10 and 4 at twelth level, and 12 and 5 at seventeenth.

\textbf{Determining Available Maneuvers.} Probably the easiest way to generate your available maneuvers each round is to take six cards out of a regular poker deck (more for higher level Crusaders) and write the card symbols next to the maneuvers you want to ready. Then each turn shuffle the cards you had last turn into the cards you didn't have and deal yourself 2 new cards (more for higher level Crusaders). More enterprising players might want to write the actual names of their maneuvers onto cards and then prepare a new "deck" when they ready different maneuvers. A player could also use dice: write down a number next to each readied maneuver and then roll 2d6 (or 3d8, 4d10, or 5d12 for higher level Crusaders), with repeated numbers being rerolled until all dice are unique. The method you use isn't terribly important, so long as each maneuever has an equal chance of appearing each round.

\textbf{Stances:} A Stance is like a Boost that the Crusader can always use regardless of what maneuvers are ready or available. Each round they may expend a Swift Action to enact one of their known stances for one round. A Crusader knows one Stance at first level and learns a new one every three levels after that (levels 4, 7, 10, 13, 16, and 19). However, whenever a Crusader gains a level of any class, they may forget one of the Stances they know and learn a new Stance that they qualify for.

\textbf{Furious Counterstrike (Ex):} A Crusader has one Immediate Action and one Swift Action each turn instead of one or the other.

\textbf{Steely Resolve (Ex):} At 2nd level, a Crusader gains fast healing equal to half their character level when they are at less than half hit points.

\textbf{Banner of Conviction (Ex):} Allies within line of sight of a 2nd level Crusader gain a +2 Morale Bonus on saves versus Fear and to Morale tests.

\textbf{Indomitable Soul (Ex):} At 3rd level, a Crusader adds their Charisma modifier (if positive) to all Saving Throws.

\textbf{Mettle (Ex):} If a 4th level Crusader succeeds at a saving throw against an effect that is Fortitude or Will partial, the effect is negated as if the line read "Fortitude Negates" or "Will Negates".

\textbf{Zealous Surge (Ex):} A 5th level Crusader can reroll one Saving Throw per day. This does not require an action.

\textbf{Smite (Ex):} A 6th level Crusader can declare a "Smite" and gain an extra Swift Action (perhaps to use a Boost and a Stance, or two Stances in a turn or whatever). The Smite can only be declared once each until the next time the character readies maneuvers (though they need not actually change which maneuvers they are readying).

\textbf{Interrupting Fury (Ex):} A 6th level Crusader adds a bonus equal to their Charisma modifier to attack rolls made on other character's turns. The bonus is gained whether the attack is made because of an attack of opportunity, the effects of a Counter, a readied action, or any other effect that causes the Crusader to get an attack during another character's turn.

\textbf{Never Give Up (Ex):} An 8th level Crusader is not rendered unconscious by having more Subdual Damage than they have remaining hit points. Among other things, this means that they remain conscious while below zero hit points.

\textbf{Very Persuasive (Ex):} A 9th level Crusader gains a bonus to Intimidate checks equal to their ranks in the Intimidate skill. They also gain a +3 bonus to their Leadership score (if any).

\textbf{Never Surrender (Ex):} A 10th level Crusader is immune to [Compulsion] and [Fear] effects.

\textbf{So Much Smiting (Ex):} An 11th level Crusader may use their Smite every round, essentially giving them 2 Swift Actions per turn all the time.

\textbf{Landlord:} At 12th level, a crusader gains \linkfeat{Landlord} as a bonus feat.

\textbf{So Much Zealotry (Ex):} A 13th level Crusader regains the use of their Zealous Surge every time they ready maneuvers.

\textbf{Improved Mettle (Ex):} If a 14th level Crusader fails a Fortitude Partial or Will Partial save, they are affected with the partial effect as if a character without Mettle had succeeded at the save.

\textbf{Keep Fighting (Ex):} A 15th level Crusader can extend the duration of their stances or boosts from one round to two rounds. They may trigger this ability a number of times equal to their Charisma modifier each time they ready maneuvers. Extending a Boost or Stance is not an action.

\textbf{Leadership:} At 16th level, a Crusader gains a [Leadership] feat as a bonus feat.

\textbf{Weapon Focus:} At 18th level, the Crusader gains Weapon Focus and Weapon Specialization as bonus feats.

\textbf{World Conquest (Ex):} At 20th level, a Crusader wins the game.

\subsubsection{Names of Crusader Techniques}

Techniques of Crusaders, whether they are stances or maneuvers, have individual names. These are generally procedurally generated as described below and contain a "unique bit". The unique bit for a technique could be anything, but established schools of Crusading have terms that they use over and over again. Below are pieces of text commonly used by several schools as well as some "generic" pieces of text you might put into your technique names regardless of what school you belong to (if any).

\begin{smallbasictable}{Crusader Technique Names}{l l l l l l}
\textbf{White Raven} & \textbf{Devoted Dragon} & \textbf{Stone Spirit} & \textbf{Storm Guardian} & \textbf{Amethyst Mind} & \textbf{Generic}\\
War & Righteous & Rock & Rain & Diamond & Eastern/East\\
Wings & Golden Wyvern & Cavern & Hurricane & Ruby & Western/West\\
Victory & Guardian & Hope & Wind & Sapphire & Northern/North\\
Covering & Dragon & Piercer & Tumult & Glass & Southern/South\\
Gambit & Flame & Mantle & Mountain & Obsidian & (Character's Name)\\
Leader & Fiery & Trapper & Dark cloud & Blue & Elothar\\
Tactics & Faith & Lurker & \textbf{bingo} & Emerald & Perfect\\
White Raven & Metal & Loyalty & Wave & Topaz & Justice\\
White & Burning & Hammer & Depression & Singular & Rage\\
Raven & Drake & Column & Cyclone & Quandary & Wings\\
Eponymous & Scales & Chasm & Gust & Shatter & Progress\\
Self Referent & White & Spectral & Tornado & Safety & Strength\\
Inspirational & Red & Gloom & Water & Inspired & Power\\
Wolf & Green & Despair & Fan Dance & Cool & Tiger\\
Skull & Blue & Soul & Storm & Purple & Rhino\\
Strategem & Black & Darkness & Wash & Cutter & Mouse\\
\end{smallbasictable}

%%%
\subsubsection{Crusader Maneuvers}
%%%

A Crusader's maneuvers can follow several naming conventions, which broadly fall into the category of "[Unique Bit] [Table Entry]" and "[Table Entry] [Unique Bit]". However, they can be tied together with possessives or prepositions in any way that makes them seem to flow better. Common choices for connectors are things like "of the". So by choosing the table entry "Nightmare Blade" and the unique bit "Dragon", you could plausibly have "The Dragon Nightmare Blade", "Dragon's Nightmare Blade", "Nightmare Blade Dragon", "Nightmare Blade of Dragon", or "Nightmare Blade of the Dragon". You can also experiment with converting nouns to adjectival forms to make things like "Draconic Nightmare Blade". Really: go nuts. Each maneuver can have its own unique bit or not. Use "Gatotsu" for everything or use a different color for each one, both can be cool.

\newcommand{\crusadermaneuver}[4]{\textbf{#1} [Level #2 #3] #4\medskip{}}

\crusadermaneuver{Bone Crusher}{1}{Strike}{The Crusader makes a melee attack. This attack ignores any hardness, Energy Resistance, or Damage Reduction the target may have.}

\crusadermaneuver{Brutal Strike}{1}{Strike}{The Crusader hits someone super hard with a melee attack. The attack does an extra d6 of damage. This bonus increases to 2d6 at 3rd level, 3d6 at 4th level, 4d6 at 6th level, 5d6 at 7th level, 7d6 at 8th level, 9d6 at 9th level, 11d6 at 10th level, 14d6 at 11th level, 16d6 at 12th level, 20d6 at 13th level, 100 points at 14th level, 110 points at 15th level, 125 points at 16th level, 150 points at 17th level, 175 points at 18th level, 200 points at 19th level, and 250 points of damage at 20th level.}

\crusadermaneuver{Charging Minotaur}{1}{Strike}{The Crusader moves their speed (minimum 10 feet) into an opponent's square and initiates a Bull Rush (with the usual +2 for charging). If this succeeds, the character gains an immediate free attack with a +4 circumstance bonus on the to-hit roll.}

\crusadermaneuver{Entrapment of Blades}{1}{Counter}{Can be used when a threatened opponent moves (including to take a 5' step). The Crusader may make a melee attack against that target.}

\crusadermaneuver{Leading the Attack}{1}{Strike}{The Crusader makes a melee attack. If it inflicts damage on the target, the target provokes an attack of opportunity from every other creature.}

\crusadermaneuver{Perfect Moment}{1}{Counter}{The Crusader may make a skill check instead of a Saving Throw. The character may choose the skill used.}

\crusadermaneuver{Nightmare Blade}{1}{Strike}{The Crusader may make a normal melee attack. The target is treated as flat footed. The limit to how much attack bonus can be sacrificed into Power Attack while making a Nightmare Blade Strike is increased by 10.}

\crusadermaneuver{Second Wind}{1}{Strike}{The Crusader makes a melee attack. If it damages the target, and the Crusader is at less than half hit points, they are healed enough to bring them to half hit points.}

\crusadermaneuver{Stone Bones}{1}{Boost}{The Crusader gains DR X/Stone, where X is 3 + Level.}

\crusadermaneuver{Taunting Blow}{1}{Strike}{The Crusader makes a melee attack. If it damages the target, and the Crusader has less temporary hit points than their Charisma modifier, they now have as many temporary hit points as their Charisma modifier. These temporary hit points expire the next time the character readies maneuvers.}

\crusadermaneuver{Boulder Roll}{2}{Strike}{The Crusader may move their full speed and then make a melee attack. If the attack does any damage, the target is pushed back 5' and falls prone.}

\crusadermaneuver{Blinding Blow}{2}{Strike}{The Crusader makes a melee attack. If the target takes any damage, they must make a Fortitude save or be blinded.}

\crusadermaneuver{Revengeance}{2}{Counter}{Use when the Crusader is damaged by an opponent. The Crusader immediately responds by making a melee attack against that opponent.}

\crusadermaneuver{Vision of Victory}{2}{Boost}{The Crusader gains a +10 Insight bonus to their first attack while this Boost is active, and that attack ignores concealment.}

\crusadermaneuver{Executioner}{3}{Strike}{The Crusader makes a melee attack. If it inflicts any damage, the target must make a Fortitude Save or die.}

\crusadermaneuver{Oldest Ploy}{3}{Strike}{All threatened enemies must make a Reflex Save or be blinded for one round. If at least one enemy is unable to see, the Crusader may make a melee attack against one of them.}

\crusadermaneuver{Scary Face}{3}{Boost}{The Crusader may take one action with the Intimidate skill that would normally require a standard action (such as demoralize opponent) as part of initiating this Boost.}

\crusadermaneuver{Toppling Tower Touch}{3}{Counter}{Use when a threatened opponent moves (including 5' steps). The Crusader makes a melee attack against that target. If it inflicts damage, the Crusader may also make a free Trip attempt.}

\crusadermaneuver{Whirlwind}{3}{Strike}{The Crusader makes a melee attack against each target they threaten.}

\crusadermaneuver{Foehammer}{4}{Strike}{The Crusader makes a melee attack. This attack does an extra 3d6 of damage and ignores any hardness, damage reduction, or energy resistance of the target. If the target takes any damage, they are staggered for one round.}

\crusadermaneuver{Lingering Strike}{4}{Strike}{The Crusader hits the target with a cruel and lingering wound that disrupts their concentration. The Crusader makes a standard attack (melee or ranged). The attack does double damage, and is considered to be ongoing damage for purposes of Concentration checks made in the following round.}

\crusadermaneuver{Listen Carefully}{4}{Boost}{The Crusader "sees" invisible and incorporeal things and ignores all concealment. The Crusader gains a +4 Insight bonus to attacks and Armor Class. This Boost does not function if the Crusader is deaf.}

\crusadermaneuver{Wraith Strike}{4}{Strike}{The Crusader makes a melee attack. This attack is resolved as a melee touch attack.}

\crusadermaneuver{Hydra Strike}{5}{Strike}{The Crusader makes a number of basic melee attacks at their highest attack bonus equal to their Charisma modifier. No more than 2 of these attacks can be directed to a single target.}

\crusadermaneuver{Demoralizing Strike}{5}{Strike}{The Crusader hits a target so ugly that it freaks out all of their allies. The Crusader makes a melee attack. If the target takes any damage they are sickened for the next minute. All enemies within line of sight must make a Will save or become shakened for one round.}

\crusadermaneuver{Subtlest Defense}{5}{Counter}{Use when the Crusader is targeted by an attack or magical effect. The Crusader may immediately take a five foot step. If the attack is no longer capable of hitting the Crusader, it fails.}

\crusadermaneuver{Psychup}{5}{Boost}{The character adds a d12 to all their damage rolls.}

\crusadermaneuver{Bonesplitting Strike}{6}{Strike}{The Crusader makes a melee attack. If the target suffers any damage, the target also takes 2d6 of Constitution damage.}

\crusadermaneuver{Heart Seeker}{6}{Strike}{The Crusader makes an attack (melee or ranged), and if that attack does any damage, the target must make a Fortitude save or die.}

\crusadermaneuver{Serpent and Mongoose}{6}{Strike}{The Crusader and every ally within short range may make an attack (melee or ranged).}

\crusadermaneuver{Wizard Ender}{6}{Counter}{When a target the Crusader can see attempts to cast a spell or use a spell-like ability, the Crusader can make an attack (melee or ranged) against that target. If the attack inflicts damage, the target must make a concentration check or lose their spell as normal.}

\crusadermaneuver{Covering Fire}{7}{Counter}{Use when an ally moves. The Crusader may make a full attack (melee or ranged), and the ally provokes no attacks of opportunity for moving through the threatened areas of any opponents the Crusader attacks with this action.}

\crusadermaneuver{Critical Strike}{7}{Strike}{The Crusader makes an attack (melee or ranged), and if it hits it is a critical as if it had rolled a natural 20 and then confirmed. This attack ignores Cover less than Total Cover and bypasses all Damage Reduction.}

\crusadermaneuver{Invisible Path Following}{7}{Counter}{Use after another character uses a teleport effect that either begins or ends within long range of the Crusader. The Crusader immediately teleports to a point adjacent to where the originally teleporting character ends up (regardless of plane or distance) and then may make a normal melee attack.}

\crusadermaneuver{Overwhelming Mountain}{7}{Strike}{The Crusader makes a melee attack. If the target takes any damage, they are stunned for one round.}

\crusadermaneuver{Backstroke}{8}{Counter}{This maneuver can be used to interrupt any action. The Crusader makes an attack (melee or ranged), which may force a concentration check if it interrupts a spell or spell-like ability.}

\crusadermaneuver{Demon Ender}{8}{Strike}{The Crusader makes an attack (melee or ranged). If the target takes any damage, they are returned to their plane of origin and are dimensional anchored for one day.}

\crusadermaneuver{Forrest of Blades}{8}{Strike}{The Crusader may move their speed and make one melee attack against each opponent they threaten at any portion of their move. If any opponent takes an attack of opportunity against the Crusader, the Crusader may make a second attack against that opponent.}

\crusadermaneuver{Pressing Attack}{8}{Boost}{The Crusader adds 40 to his second damage roll this turn.}

\crusadermaneuver{Beast Killer}{9}{Strike}{The Crusader makes two attacks (melee or ranged) at the same or different targets. Each attack inflicts an extra 3 points of Intelligence and Wisdom damage.}

\crusadermaneuver{Man Slayer}{9}{Strike}{The Crusader makes an attack (melee or ranged). If this attack strikes an opponent who is neither larger than the Crusader nor has more hit dice than the Crusader, that target takes enough damage to kill them instantly. Otherwise, the target takes normal damage.}

\crusadermaneuver{Shield Counter}{9}{Counter}{This may be used to interrupt any action. The Crusader makes a melee attack with their shield. If this attack does any damage, the target's interrupted action (if any) is canceled and their turn ends. The Crusader must actually have a shield to use this, but they do not lose their shield bonus to armor class for making a shield bash attack.}

\crusadermaneuver{Soulcrusher}{9}{Strike}{The Crusader makes an attack (melee or ranged). This attack does double damage. If the target is dead after being damaged by this attack (whether or not it was dead before hand), it cannot be brought back from the dead (as living or undead) by anything short of true resurrection.}

\crusadermaneuver{Earthshaker}{10}{Strike}{The Crusader makes a melee attack that does double damage. All creatures within short range (other than the Crusader) that are touching the ground must make a Fortitude Save or fall prone.}

\crusadermaneuver{Iron Bones}{10}{Boost}{The Crusader gains DR of X/Iron, where X is 5+Level.}

\crusadermaneuver{Pain}{10}{Strike}{The Crusader makes an attack (melee or ranged). If the target suffers any damage, they suffer a penalty on saving throws and skill checks equal to the amount of damage inflicted for one minute.}

\crusadermaneuver{Revival Strike}{10}{Strike}{The Crusader makes an attack (melee or ranged). If the target suffers any damage, then an ally within 5' of the target is healed enough to bring them to half hit points. This is a Supernatural Ability.}

\crusadermaneuver{Golem Ender}{11}{Strike}{The Crusader makes an attack (melee or ranged). If the target is hit, it takes double damage. If the target is damaged and it is a Construct, it is destroyed.}

\crusadermaneuver{Immortal Fortitude}{11}{Boost}{If the Crusader would die (whether from hit point damage or some other effect) before the end of their next turn, they will not die until the end of their next turn. If they heal enough to not die, or gain immunity to whatever effect would render them dead in the interim, they do not in fact die at that point.}

\crusadermaneuver{Manticore Parry}{11}{Counter}{Use when the Crusader is targeted by an attack. That attack misses the Crusader, the Crusader then may choose a new target that is within range of that attack and the attack is resolved normally as if the new target was the original target.}

\crusadermaneuver{Tide of Battle}{11}{Boost}{Next turn, instead of randomly determining all of your available maneuvers, you may choose 2 of them and randomly determine the rest.}

\crusadermaneuver{Rallying Strike}{12}{Strike}{The Crusader makes an attack (melee or ranged), and if it successfully does damage, the Crusader may end any number of ongoing [Mind Affecting] or [Fear] effects.}

\crusadermaneuver{Elder Mountain}{12}{Counter}{Use when an opponent attacks or uses a spell or spell-like ability. The Crusader interrupts that action with an attack (melee or ranged). This attack bypasses any DR, Hardness, or Energy Resistance of the target. The attack also gains a +3d6 bonus to the damage roll.}

\crusadermaneuver{Giant Size}{12}{Boost}{The Crusader and their equipment is counted as one size larger than they are for purposes of reach, special combat maneuvers, and weapon damage. They aren't actually a different size and are not penalized for squeezing or attack rolls.}

\crusadermaneuver{Zebra Tone}{12}{Boost}{The Crusader gains a +8 Luck bonus to AC, attack, and damage rolls.}

\crusadermaneuver{Aura of Tyranny}{13}{Boost}{Creatures within line of sight of the Crusader that have less hit dice than the Crusader are shaken until the Aura expires.}

\crusadermaneuver{Beryl Defense}{13}{Counter}{Use just before making a Saving Throw. The Crusader may add their Base Attack Bonus to that Saving Throw.}

\crusadermaneuver{Righteous Vitality}{13}{Strike}{The Crusader makes an attack (melee or ranged). If that attack does any damage, the Crusader is affected by heal (caster level equals the Crusader's level). This is a Supernatural ability.}

\crusadermaneuver{Crushing Vice}{13}{Strike}{The Crusader makes a melee attack that does double damage. If the target suffers any damage, they are also pinned until next turn.}

\crusadermaneuver{Dazing Strike}{14}{Strike}{The Crusader makes an attack (melee or ranged). The attack does triple damage. If any damage is inflicted on the target, the target is dazed for d3 turns.}

\crusadermaneuver{Puff of Smoke}{14}{Counter}{Use in response to any action. The Crusader interrupts the action and moves their speed without provoking an attack of opportunity. If the action being interrupted is no longer legal, it fails. If the action being interrupted would now provoke an attack of opportunity from the Crusader, the Crusader is entitled to that attack of opportunity.}

\crusadermaneuver{Breath of Fire}{14}{Strike}{The Crusader breathes a gout of flame. All targets in a short range cone suffer a d8 of fire damage per level. Reflex Saves halve the damage. This is a supernatural ability.}

\crusadermaneuver{Shadows and Moonbeams}{14}{Boost}{The character can fly at double their normal ground speed with perfect maneuverability. This is a supernatural ability. This Boost lasts until the next time the character initiates a Boost or Stance.}

\crusadermaneuver{Castigations}{15}{Strike}{The Crusader pounds their hands together and a great explosion erupts. All creatures (other than the Crusader) within short range of the Crusader must deal with Hurricane Force Winds and take a d6 per level of Sonic damage (Fortitude Partial for half damage and no wind check).}

\crusadermaneuver{Dispersal}{15}{Strike}{All summoned creatures and objects within medium range of the Crusader vanish and do not return.}

\crusadermaneuver{Manifest Destiny}{15}{Counter}{Use when a d20 is rolled, after seeing the result. Roll another d20 and choose which die result is used. The Crusader may use this to interfere in the rolls of other characters, whether friend or foe.}

\crusadermaneuver{Quicksilver Motion}{15}{Boost}{The Crusader takes an extra move action this turn.}

\crusadermaneuver{Avalanche of Blades}{16}{Strike}{The Crusader makes a number of attacks (melee or ranged) equal to their level.}

\crusadermaneuver{Restoration Strike}{16}{Strike}{The Crusader makes an attack (melee or ranged). If the target suffers any damage, an ally within 5' of the target is healed of ability damage and drain and negative energy levels and such as per greater restoration. This is a supernatural ability.}

\crusadermaneuver{Steel Bones}{16}{Boost}{The Crusader has DR of X/Adamantine, where X is 10 + Level. The Crusader also as SR of 10 + Level.}

\crusadermaneuver{Vanishing Defense}{16}{Counter}{The Crusader disappears, and is invisible until the end of their next turn.}

\crusadermaneuver{Banishing Strike}{17}{Strike}{The Crusader smashes the target out of existence. The Crusader makes a standard attack (melee or ranged), and if the target suffers any damage it is sent to another plane of the Crusader's choice, where it is then dimensional anchored for a day. The Crusader may travel with the target if they choose.}

\crusadermaneuver{Hatred and Malice}{17}{Boost}{The Crusader gains a +50 bonus on damage rolls.}

\crusadermaneuver{Incorporeal Self}{17}{Boost}{The Crusader is incorporeal. Their attacks also damage incorporeal and ethereal things.}

\crusadermaneuver{Murder Vision}{17}{Strike}{The Crusader makes a melee attack against a target that does not have to be within their reach. The melee attack reaches out to line of sight if necessary and does quadruple damage.}

\crusadermaneuver{Denial}{18}{Counter}{Use when an action is declared within long range of the Crusader. That action is canceled.}

\crusadermaneuver{Five Shadow Creeping Enervation of Icy Darkness and Seering Light Strike}{18}{Strike}{The Crusader makes a melee or ranged attack. If the target is struck (whether they are damaged or not), the Crusader may choose three conditions and inflict them on the target. Yes, that includes petrified and dead if desired.}

\crusadermaneuver{Singularity}{18}{Boost}{The Crusader's speed is increased by 200'. During the turn, the character can walk on water, paper, or air without leaving a trail or being in danger of falling. The Crusader may take their Standard Action at any point during their movement and continue moving afterward.}

\crusadermaneuver{Ultima Bladerush}{18}{Strike}{The Crusader moves their speed and makes a melee attack against any opponent that comes into their threatened range at any point during this movement. This movement does not provoke attacks of opportunity. Each attack inflicts double damage and bypasses Damage Reduction, Hardness, Energy Resistance, and Regeneration.}

\crusadermaneuver{Unstoppable Blow}{19}{Strike}{The Crusader inflicts damage as if they inflicted a critical hit with whatever weapon they are using. No to-hit rolls, miss chances, or saving throws are checked.}

\crusadermaneuver{Golem Soul}{19}{Boost}{The Crusader gains Spell Immunity. It's just like they had Spell Resistance, but it cannot be overcome.}

\crusadermaneuver{Time Stopper}{19}{Counter}{The Crusader interrupts whatever is happening and takes a complete turn (though they do not draw new maneuvers), then the turn order goes back to proceeding as normal.}

\crusadermaneuver{World Ender}{19}{Strike}{The Crusader makes an attack (melee or ranged). If it does any damage to the target, that target is destroyed.}

%%%
\subsubsection{Crusader Stances}
%%%

All stances are essentially Boosts in that they take a Swift Action to enact and last until the beginning of the Crusader's next turn. Each Crusader should name their stances, with the nomenclature of "[Unique Bit] [Ability Entry] Stance". So, for example, a Crusader might choose as their unique bit "Burning" and the ability entry "Clarity" and have "Burning Clarity Stance". Each stance can have its own unique bit or not.

\newcommand{\crusaderstance}[3]{\textbf{#1} [Level #2] #3\medskip{}}

\crusaderstance{Clarity}{1}{Choose a target. The Crusader gains +2 to AC and Reflex Saves against attacks and abilities from that target.}

\crusaderstance{Imposing}{1}{The Crusader may make one extra Attack of Opportunity each turn.}

\crusaderstance{Punishment}{1}{The Crusader inflicts +1d6 of damage, but has -2 AC.}

\crusaderstance{Earthroot}{1}{The character gains +20 to resist trips, bullrushes, and overruns. And +2 to Fortitude saves.}

\crusaderstance{Mind Prism}{1}{The character gains a +20 to resist disarms and feints. And a +2 to Will saves.}

\crusaderstance{Spinning}{2}{The character cannot be flanked and gains a +2 bonus to Reflex saves.}

\crusaderstance{Dragon Slaying}{2}{The character gains +2 to attack and damage rolls against enemies that have more hit dice than they do.}

\crusaderstance{Ghost Fighting}{3}{The character can touch and strike ethereal and incorporeal creatures with their melee attacks without suffering a miss chance for doing so.}

\crusaderstance{Light Stepping}{3}{The character does not exert noticeable pressure on the ground and does not leave tracks. They can also stand on and move across water without falling in. This stance lasts until the character activates a different stance.}

\crusaderstance{Aggressive}{4}{The character gains a bonus to damage rolls equal to their BAB.}

\crusaderstance{Magicguarding}{4}{The character has Spell Resistance of 5 + Level.}

\crusaderstance{Fortification}{5}{The character is immune to critical hits.}

\crusaderstance{Giant Slaying}{5}{The character gains a +4 bonus to AC and attack rolls against enemies that are larger than they are.}

\crusaderstance{Mountain Roots}{6}{The character gains a +10 bonus to resist Trip, Bullrush, Grapple, and Overrun. The character gains a +4 bonus to initiating any of these actions themselves.}

\crusaderstance{Fast}{6}{The character's movement rate is increased by 30 feet.}

\crusaderstance{Hungry}{7}{The character's attacks ignore concealment.}

\crusaderstance{Shattered Soul}{7}{The character is considered to have a different alignment than their actual alignment for purposes of spells and effects. The alternate alignment is chosen when the stance is activated.}

\crusaderstance{Anchored}{8}{The character cannot be teleported, summoned, or shifted into another plane. If they pass though a portal while in this stance, they simply move to the other side on the same plane of existence. This stance lasts until the character activates a new stance.}

\crusaderstance{Eager}{8}{The character adds +4 to initiative checks. This stance can be declared before making an Initiative check on the first round, but it still uses up a Swift Action on the first round of combat.}

\crusaderstance{Air Walking}{9}{The character does not fall, meaning that they are able to stand and move normally in the middle of the air at whatever height they have already achieved. They can also perform a series of high jumps to get to higher elevations. When the stance expires, the character loses 5 feet of elevation before they can reinitiate it.}

\crusaderstance{Very Fast}{9}{The character's movement rate is increased by 60 feet.}

\crusaderstance{Wolf Pack}{10}{The character can take an extra 5' step every time one of their attacks inflicts damage. This extra 5' step may be taken even if the character used normal movement and would not normally be allowed to take a 5' step that turn.}

\crusaderstance{Troll Slaying}{10}{The character's attacks do normal damage and ignore reductions or transformations from Damage Reduction, Regeneration, Hardness, or Energy Resistance.}

\crusaderstance{Alacritous}{11}{The character can take one extra Immediate Action this coming turn.}

\crusaderstance{Cautious}{11}{Whenever the character rolls a saving throw, they may roll two d20s and select the result they like better.}

\crusaderstance{Swarming}{12}{All allies gain a +5 morale bonus on attack rolls against enemies you threaten.}

\crusaderstance{Tumultuous}{12}{Whenever the character rolls an attack roll, they may roll two d20s and select the result they like better.}

\crusaderstance{Lucky}{13}{While in this stance, enemies within short range have a -2 Luck penalty on their Saves, Attack Rolls, and Checks.}

\crusaderstance{Stunning}{13}{Any time the character inflicts damage on a target, that target must make a Fortitude Save or be stunned for one round.}

\crusaderstance{Murderous}{14}{The first damage roll the Crusader makes each turn gains a bonus equal to twice the Crusader's BAB.}

\crusaderstance{Chocolaty}{14}{The character is immune to ability damage and ability drain.}

\crusaderstance{Selfless}{15}{The character is immune to [Mind Affecting] effects and does not appear in divinations. Actions taken while in this stance are considered to not happen for the purposes of the answers divinations give. This stance lasts until the character activates a new stance.}

\crusaderstance{Prophetic}{15}{The character is not flat footed and acts normally this round. This stance can be activated before the character's first turn in a battle, but it still uses up a Swift Action in the first turn of that battle.}

\crusaderstance{Earth Tapping}{16}{The character's speed is reduced by 10. So long as they are touching the ground, they have Regeneration 5/Electricity or Epic.}

\crusaderstance{Careful}{16}{The character can take 10 on any d20 roll, even rolls that are not checks.}

\crusaderstance{Pointless}{17}{The character does not look abnormal or out of place anywhere they go, even places where literally no one is normally allowed to be. Unless the Crusader actually engages in combat actions, their presence prompts not a second glance.}

\crusaderstance{Soulless}{17}{If the character dies while in this stance, they return to life the next day unless their head has been removed from their body in the meantime.}

\crusaderstance{Insulting}{18}{Enemies within short range of the Crusader are incapable of attacking or targeting any of the Crusader's allies unless the same action also targets or attacks the Crusader. A character can attempt to overcome this restriction for one round by spending a Swift Action to make a Willpower Save.}

\crusaderstance{Invulnerable}{18}{Spell effects of 4th level or less do not function within short range of the character. As per globe of invulnerability except as described.}

\crusaderstance{Dreadful}{19}{All enemies within line of sight of less than 8 hit dice are panicked. This is a [Mind Affecting] [Fear] effect.}

\crusaderstance{Victorious}{19}{Whenever the character rolls a d20, they may roll two d20s and choose which one they like better.}
