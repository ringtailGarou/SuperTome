%%%%%%%%%%%%%%%%%%%%%%%%%%%%%%%%%%%%%%%%%%%%%%%%%%
\raceentry{Orc}
%%%%%%%%%%%%%%%%%%%%%%%%%%%%%%%%%%%%%%%%%%%%%%%%%%
\tagline{"Waaarrrggghhhh!"}

Orcs get the short end of the stick. They can eat pretty much anything and they have to because their race has lost every major war since\ldots{} well \textit{forever}. Orcs are extremely specialized, and rarely see play as anything except a Barbarian. However, some players will want to diversify the concept into say\ldots{} a Rogue, Assassin, or Fighter build. That works okay, but remember that an Orc always brings "hitting things really hard" to the party. The Orcs other limitations are pretty severe, so taking a class combination that doesn't accentuate the narrow scope of Orc advantages is probably a mistake in the long run.

\begin{itemize*}
\item Medium Size
\item 30ft movement.
\item Humanoid Type (Orc subtype)
\item Darkvision 60ft
\item +4 Strength, -2 Intelligence, -2 Charisma, -2 Wisdom
\item Daylight Sensitivity: While in brightly lit surroundings (such as a \textit{daylight} 
spell), an Orc suffers the \textit{dazzled} condition and is thus at a -1 penalty 
to attack rolls and precision-based skill checks.
\item +2 racial bonus to saving throws vs. Poison and Disease.
\item Immunity to ingested poisons.
\item +2 to \linkskill{Jump} and \linkskill{Survival} checks.
\item Favored Classes: Barbarian and Cleric
\item Automatic Languages: Orc, Common
\item Bonus Languages: Dwarvish, Elvish, Giant, Gnoll, Goblin, Sylvan, Undercommon.
\end{itemize*}
